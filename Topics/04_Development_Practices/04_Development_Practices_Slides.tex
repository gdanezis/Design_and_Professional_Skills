\documentclass{beamer} % [aspectratio=169]
\usetheme{ucl}
\setbeamercolor{banner}{bg=brightblue}
\setbeamersize{description width=2em}
\setbeamertemplate{navigation symbols}{\vspace{-2ex}} 
\usepackage{soul}

%\usepackage{fontspec}
\usepackage[utf8]{inputenc}
% \usepackage[english, greek]{babel}


\usepackage[T1]{fontenc} % Turn £ into $
\usepackage{minted}
\usemintedstyle{emacs}

\usepackage{fancyvrb}
\usepackage{xcolor}
\usepackage{url}

\usepackage{natbib}
\usepackage{bibentry}
\usepackage{url}


\usepackage{tikz}
\usetikzlibrary{positioning}
\usetikzlibrary{calc,shapes.multipart,chains,arrows}
\usetikzlibrary{shapes,snakes}

\tikzset{
  treenode/.style = {align=center, inner sep=0pt, text centered,
    font=\sffamily},
  arn_n/.style = {treenode, circle, draw=black,
     text width=1.5em},% arbre rouge noir, noeud noir
  arn_r/.style = {treenode, circle, red, draw=red, 
    text width=1.5em, very thick},% arbre rouge noir, noeud rouge
  arn_d/.style = {treenode, star, star points=10, red, draw=red, 
    text width=1.5em, very thick},% arbre rouge noir, noeud rouge
  arn_x/.style = {treenode, rectangle, draw=black}% arbre rouge noir, nil
}

\newcommand\emc[1]{\textcolor{brightblue}{\textbf{#1}}}

\AtBeginSection[]{
  \begin{frame}
  \vfill
  \centering
  \begin{beamercolorbox}[sep=8pt,center,shadow=true,rounded=true]{title}
    \usebeamerfont{title}\insertsectionhead\par%
  \end{beamercolorbox}
  \vfill
  \end{frame}
}

\author{Prof.\ George Danezis, University College London, UK}
\title{Develpment Practices, Modules, Tools \& Releases.}
\subtitle{ENGS102P: Design and Professional Skills }
% \institute{}
\date{Term 1, 2017}


\begin{document}
\nobibliography*


\frame{
\titlepage
}

\begin{frame}
\frametitle{Outline}

This topic is all about \emc{scaling up} the development process, while maintaining \emc{feasibility and quality}.

\begin{itemize}
  \item Traditional software \emc{development methods \& disasters}.
  \item The \emc{Agile} software development framework.
  \item \emc{Tools} to support development teams.
  \item From \emc{modules} to \emc{libraries}.
  \item \emc{Release} management \& maintenance.
\end{itemize}

\end{frame}

\section{Traditional software development.}

\begin{frame}
\frametitle{Software development methods \& project management.}

Different than traditional engineering management:
\begin{itemize}
  \item Recent discipline, immature processes and tools \\ (\emc{incidental complexity}).
  \item \emc{Intrinsic complexity} increased through \emc{no physical constraints}.
  \item Tight integration between requirements, specification, implementation, operation -- no \emc{manufacturing} stage.
  \item Complexity increased through \emc{flexibility} -- ever changing \& evolving requirements.
  \item Difficult to \emc{estimate} time to completion \& defect rate.
  \item Orders of magnitude differences in \emc{programmer productivity}.
\end{itemize}

\end{frame}

\begin{frame}
\frametitle{Anti-pattern: The Waterfall model.}

Adapted from other engineering (construction). The \emc{Waterfall model} structures a project as a sequence of:
\begin{itemize}
\item \emc{Requirements gathering}
\item \emc{System Design \& Specification}. \\ Also known as `Big Design Up Front'.
\item \emc{Implementation}
\item \emc{Testing \& Validation}
\item \emc{Deployment \& Maintenance}
\end{itemize}

\begin{block}{}
\small Winston Royce (Director at Lockheed Software Technology Center) in 1970 presents the model as a `bad idea'. It is later (1985) formalized -- in terms of auditing steps -- as a methodology in US Dept.\ of Defense `Defense Systems Software Development' (DOD-STD-2167A). Subsequent MIL-STD-498 in 1994 states a preference for `Iterative and incremental development'.
\end{block}

\end{frame}

\begin{frame}
\frametitle{Problems with the Waterfall model.}

All activities included in the model need to take place at some point. However, a linear end sequential application is not appropriate, because:
\begin{itemize}
  \item \emc{Users do not know} or cannot formulate requirements, until they see working software (prototypes, early versions, \ldots)
  \item \emc{Designers may not foresee} all issues related to the implementation, or be able to chose between design options, without prototyping and testing.
  \item In case \emc{different people} are employed in different phases, \emc{communication breakdown} may occur; and only a \emc{fraction of the team is engaged} at any point.
\end{itemize}

Poorly adapted to \emc{`brown-field' development}, where existing software needs to be extended, rather than `green-field' new projects.

\end{frame}

\begin{frame}
\frametitle{The tar pit (Brooks 1975)}

\begin{center}
\includegraphics[scale=0.5]{assets/tar-pit}
\end{center}

{\small The Mythical Man-Month. Essays on Software Engineering by Frederick Brooks Jr. Anniversary Edition  Paperback, 322 pages Published by Addison-Wesley Pub Co in July 1995. (First edition 1975.)}

\end{frame}

\begin{frame}
\frametitle{Lessons from `The Mythical Man-Month' (1)}

Brooks was an architect on the IBM OS/360 -- delivered epically late in the 1970s.
\begin{itemize}
  \item  \emc{Brooks's law}: `\emph{Adding manpower to a late software project makes it later}'. The communication channels increase as $\mathcal{O}(n^2)$ in the number $n$ of developers.
  \item \emc{No silver Bullet}: No single development method, language or tool can reduce software dev. by an order of magnitude, since they may only reduce incidental not intrinsic complexity.
  \item \emc{The Second-System effect}: Be aware of the second version of a system, since the architects will try to include all options they did not include in the first.
\end{itemize}

\end{frame}

\begin{frame}
\frametitle{Lessons from `The Mythical Man-Month' (2)}

\begin{itemize}
  \item \emc{The Pilot system}: when designing a new system, plan for an initial pilot (prototype) which will help you learn. Throw it away, and deliver the second one.
  \item \emc{Project estimation}: programming products take much longer to build, than internal tools; a lot of time is taken by communications, `stand-ups' and `all-hands'.
  \item \emc{The surgical team}: Build small teams around and supporting experienced programmers.
  \item \emc{Code freeze}: freeze features at some point, and only increase the quality of code, until next release.
\end{itemize}

\end{frame}

\begin{frame}
\frametitle{Spiral model to manage risk.}

\begin{center}
\includegraphics[scale=0.27]{assets/spiral}
\end{center}

{\small Boehm ``A Spiral Model of Software Development and Enhancement'', ACM SIGSOFT Software Engineering Notes, ACM, 11(4):14-24, August 1986}

\end{frame}

\section{Agile development and practices.}

\begin{frame}
\frametitle{The Manifesto for Agile Software Development (2001).}

Radical: Re-focus on \emc{software as working code}: `The need for an \emc{alternative to documentation driven, heavyweight} software development processes.'

\vspace{5mm}
Why the crisis? Traditional software \emc{project failures} in 1995:
\begin{itemize}
  \item 16.2\% -- Project Success: on time, budget, features.
  \item 52.7\% -- Project Challenged: overrun time, budget, fewer features.
  \item 31.1\% -- Project Impaired: canceled at some point.
\end{itemize}
 (Standish Group, CHAOS Report, 1995)

\end{frame}

\begin{frame}
\frametitle{The Manifesto (1--6).}


\begin{itemize}
\item Our highest priority is to satisfy the customer
through \emc{early and continuous delivery
of valuable software}.

\item Welcome \emc{changing requirements, even late} in 
development. 

\item Deliver \emc{working software} frequently, from \emc{a 
couple of weeks to a couple of months}, with a 
preference to the shorter timescale.

\item \emc{Business people} and \emc{developers must work 
together} daily throughout the project.

\item Build projects around \emc{motivated individuals}. 
Give them the support they need, 
and \emc{trust them} to get the job done.

\item The most efficient and effective method of 
conveying information to and within a development 
team is \emc{face-to-face conversation}.

\end{itemize}

\end{frame}

\begin{frame}
\frametitle{The Manifesto (7--12).}

\begin{itemize}


\item \emc{Working software is the primary measure of progress}.

\item Agile processes promote \emc{sustainable development}. 
The sponsors, developers, and users should be able 
to maintain a \emc{constant pace indefinitely}.

\item Continuous attention to \emc{technical excellence 
and good design enhances agility}.

\item Simplicity--the art of \emc{maximizing the amount 
of work not done}--is essential.

\item The best architectures, requirements, and designs 
emerge from \emc{self-organizing teams}.

\item At regular intervals, the team \emc{reflects on how 
to become more effective}, then tunes and adjusts 
its behavior accordingly.

\end{itemize}

\end{frame}


\begin{frame}

\frametitle{Composition of an Agile team}

\begin{itemize}
\item \emc{The product manager / owner} -- Maintain and promote the product vision. Part management, part people coordination, part marketing.

\item \emc{On-site customers} -- team members representing customers, continuously capturing and discussing requirements, and discussing with programmers. \\ Can be business people. Incl. 2 per 3 programmers!

\item \emc{Programmers} -- Bulk of the team, no distinction between `architects', `developers', `testers'. Incl.\ one senior, hands-on, person and overall teams of 4--9 programmers. Although some members may have a focus.

\item \emc{Domain experts, Interaction Designers, Security, Design, and Business Analysts} -- experts in their respective domains, are on-call to help on-demand. Work with both users and team.

\end{itemize}

\end{frame}


\begin{frame}
\frametitle{Ratio of on-site customers to programmers}

Do you think the ratio is too high or too low?

\vspace{5mm}
What processes do user representatives speed up?

\end{frame}



\begin{frame}
\frametitle{The rhythm of an Agile project -- Sprints}

``You can eliminate requirements, design, and testing phases as well as the formal documents that go with them.''

\begin{center}
\includegraphics[scale=0.35]{assets/agile} \\
Agile (XP) Lifecycle. Sprints are typically 1-2 weeks.
\end{center}

{\small The Art of Agile Development by James Shore. Wiley 2007.}

\end{frame}


\begin{frame}

\frametitle{User Stories}

Stories represent \emc{self-contained, individual elements} of the project: 
\begin{itemize}
\item Individual features.
\item Typically represent one or two days of work.
\item Customer-centric, in terms of business results.
\item No implementation details, or full requirements / specifications.
\end{itemize}

\vspace{5mm}
Good user stories have the form: \\ \emc{As a \_role\_ I want to do \_action\_ in order to \_activity\_.}

\end{frame}


\begin{frame}

\frametitle{Backlogs: Product, Release \& Sprint}

All stories are stored in the \emc{product backlog}:
\begin{itemize}
  \item When a new release is planned / discussed with customers some stories are moved to the \emc{release backlog}. They are augmented by \emc{acceptance criteria}.
  \item At the start of every sprint, some stories are moved into the \emc{sprint backlog}, along with noted on \emc{how to design \& implement} them.
  \end{itemize}

Programmers are asked to `cost stories' in `days', as stories progress through backlogs:
\begin{itemize}
  \item Programmers are bad at estimating, but their estimates are \emc{off by a fixed multiplicative constant}.
  \item \emc{Velocity}: the number of story `days' the project executes (Done Done!) in a day. Base \emc{project estimation} on velocity.
\end{itemize} 

\end{frame}


\begin{frame}

\frametitle{Backlogs \& stories: just a bunch of post-its or cards}

\begin{center}
\includegraphics[scale=0.50]{assets/backlog} 
\end{center}


`Kanban' board. From \url{https://dzone.com/articles/product-backlogs-practice/}

\end{frame}


\begin{frame}

\frametitle{On-line tools for Agile project management.}

A number of tools can \emc{replace physical post-it notes}:
\begin{itemize}
  \item Trello (hip) -- \url{http://trello.com}.
  \item Github projects -- see `Projects' tab on your project page.
\end{itemize}

\vspace{5mm}
Related but different tool: \emc{Issue / Bug tracker.}
\begin{itemize}
  \item Allow users, internal and external developers to \emc{report issues}.
  \item Allows \emc{discussion} on the issue and the development of a \emc{minimal test case to reproduce the error}.
  \item Allows for \emc{triage}, and \emc{assignment} of issues to team members.
  \item Can be `abused' to report feature requests, and simulate a story board.
  \end{itemize}

\end{frame}

\begin{frame}

\frametitle{Agile practices: Pair programming}

Two programmers work together on a \emc{single workstation}. 
\begin{itemize}
\item One \emc{driver / codes} the other \emc{navigator / reviews \& thinks}.
\item They \emc{switch roles} frequently.
\end{itemize}

Why pair?
\begin{itemize}
  \item Navigator has opportunity to think about \emc{strategic design} choices. Driver can focus on  \emc{code tactics}. Higher quality code.
  \item Reinforces, and shares \emc{good programming habits}. Continuous testing and refinement result from \emc{peer pressure}.
  \item Resilient to \emc{interruptions}. When interrupted one person can handle interrupt, while other codes.
  \item Its \emc{more fun} to not work alone. Can take \emc{physical breaks}.
\end{itemize}
Focus on sustaining, rather than peaks of high production.

\end{frame}


\begin{frame}
\frametitle{Pair programming or code reviews?}

What do the practices have in common?

\vspace{5mm}
What are their advantages and disadvantages?

\end{frame}



\begin{frame}

\frametitle{Agile practices: Documentation \& Agile Modeling.}

Keep the non-code project documentation to a minimum.
\begin{itemize}
\item \emc{Document continuously}: \emc{Minimal} stories, design notes and meeting notes.
\item \emc{Document late}: Do now write anything that will not be read or acted upon. Minimize speculation.
\item \emc{Single Source Information}: All project \emc{documentation} has to be \emc{kept under version control}.
\item \emc{Prefer Executable specifications}: Write as \emc{customer tests} that can execute against the code. (TDD at the requirements level.)
\end{itemize}

\vspace{5mm}
All code and docs is kept in version control.
\end{frame}

\begin{frame}

\frametitle{Agile practices: Distributed Version Control.}

Use Distributed Version Control for \emc{fearless refactoring} and \emc{collaborative coding}:
\begin{itemize}
  \item Tool of choice: \texttt{git} and \texttt{github}.
  \item Code lives in \emc{repositories}, \emc{remote} and \emc{local}.
  \item A repository can have multiple \emc{branches}. 
  \item You may \emc{branch} and \emc{merge}, or \emc{commit} code to a branch.
\end{itemize}

\vspace{5mm}
\emc{Learn git!} \url{https://git-scm.com/documentation/external-links} 

\vspace{5mm}
Gitflow illustrations by Vincent Driessen \url{http://nvie.com}.


\end{frame}


\begin{frame}
\frametitle{Agile practices: Gitflow \& version control.}

A \emc{branch} in a git repository contains a `variant' of your repository. You can \emc{merge} a branch into another, after a number of \emc{commits}.

\vspace{5mm}
Gitflow basic organization for a repository:
\begin{itemize}
  \item The \emc{master} branch only contains tagged releases.
  \item A \emc{develop} branch is used to continuously update the software.
  \item New features are developed in separate \emc{feature} branches.
  \item A \emc{release branch} captures the features for a new release, and from there on only accepts \emc{bug fixes}.
  \item Eventually a release branch is merged into master to make a new release.
  \item \emc{Hotfixes} are applied to master and merged into develop. 
  \end{itemize}

Good programmers use it even when they work on their own!

\end{frame}

\begin{frame}

\begin{center}
\includegraphics<1>[scale=0.33]{assets/flow00} 
\includegraphics<2>[scale=0.33]{assets/flow0} 
\includegraphics<3>[scale=0.33]{assets/flow1} 
\includegraphics<4>[scale=0.33]{assets/flow2} 
\includegraphics<5>[scale=0.33]{assets/flow3} 
\end{center}

\end{frame}

\begin{frame}

\frametitle{Why different feature branches?}

Why develop different user stories on different feature branches?

\vspace{5mm}
What is the cost of diverging too much from the develop branch?

\vspace{5mm}
How do you ensure you do not diverge too much from the develop branch?

\end{frame}


\begin{frame}
\frametitle{Agile practices:`Done Done'.}

Each feature branch represents a user story, and it is merged into develop when it is `\emc{done done}.':
\begin{itemize}
  \item Fully Designed, Coded, Integrated (from UI to Database), Tested (unit, integration, and customer tests finished).
  \item Builds, Installs (incl. auto install \& devops), Migrates past state.
  \item Fixed all bugs, Reviewed by customers and accepted as finished.
\end{itemize}

\begin{block}{From working software to working software}
Executing each user story until it is `done done' takes working software and produces working software, end-to-end. This way users can see progress and provide early feedback.
\end{block}

\end{frame}

\begin{frame}
\frametitle{Agile practices:Continuous Integration.}

\begin{block}{The `build' engineer}
Merging branches that are a lot of different commits apart, is time consuming. At traditional software firms this is the job of the `build' engineer. `Integration' and testing of different branches can take months.
\end{block}

\vspace{3mm}
\emc{Continuous Integration}: be technically ready to release, even if you are not functionally ready.
\begin{itemize}
  \item Create automated pipeline for testing integration, packaging, installers, deployment in servers, and data migration.
  \item Run those tests on every commit to develop branch.
  \end{itemize}

\vspace{5mm}
Services pull from github as Travis-CI (\url{https://travis-ci.org/}).

\end{frame}

\begin{frame}

\frametitle{Why testing continuously?}

Integration, Install, Deployment, Migration: even when the features are not entirely finished or there?

\end{frame}

\begin{frame}

\begin{center}
\includegraphics<1>[scale=0.50]{assets/travis} 
\end{center}

\end{frame}


\begin{frame}

\frametitle{Other Agile practices}

Other tricks of the trade:
\begin{itemize}
  \item Test driven development.
  \item Agile testing \\ (specification by example.)
  \item Refactoring.
  \item Cross-functional teams.
  \item Information Radiators.
  \item Planning poker.
  \item Scrum events \\ (planning, daily Stand-up meetings).
  \item Timeboxing.
\end{itemize}

\end{frame}

\begin{frame}
\frametitle{Conclusion: `No silver bullet'.}

\begin{center}
\includegraphics<1>[scale=0.90]{assets/agilewaterfall} \\
{\tiny from \url{https://www.infoq.com/articles/standish-chaos-2015}}
\end{center}


\end{frame}

\section{Python packages, libraries, and deployment.}


\begin{frame}

\frametitle{Import Python libraries}


Import a library by name:
  \inputminted[
    firstline=3,
    lastline=5,
    xleftmargin=1.4em,
    frame=lines,
    framesep=2mm,
    %baselinestretch=1.2,
    % bgcolor=lightgray,
    fontsize=\scriptsize,
    linenos
  ]{python}{src/example_import.py}

Import specific names within the library:
  \inputminted[
    firstline=7,
    lastline=8,
    xleftmargin=1.4em,
    frame=lines,
    framesep=2mm,
    %baselinestretch=1.2,
    % bgcolor=lightgray,
    fontsize=\scriptsize,
    linenos
  ]{python}{src/example_import.py}

Import and rename a library:
  \inputminted[
    firstline=11,
    lastline=12,
    xleftmargin=1.4em,
    frame=lines,
    framesep=2mm,
    %baselinestretch=1.2,
    % bgcolor=lightgray,
    fontsize=\scriptsize,
    linenos
  ]{python}{src/example_import.py}

Look for library in the folders \texttt{sys.path}.

\end{frame}

\begin{frame}

\frametitle{The Python Standard Library}

The standard libraries are available with any full python installation. You can \emc{rely on them}!
\begin{itemize}
  \item \emc{Basic programming}: Built-in types, text processing, binary data, dates, calendars, maths, functional tools, \ldots
  \item \emc{System}: Files, directories, data formats, compression, cryptography, operating system functions, \ldots
  \item \emc{Communications}: Inter-process communications, internet protocols, markup, multimedia, i19n, \ldots
  \item \emc{Language}: Debug / profile, distrbute, runtime, interpreters, import, language, \ldots
  \end{itemize}

Always \emc{prefer to use the standard library} over your own types.

\end{frame}

\begin{frame}

\frametitle{Non-standard libraries}

Third party libraries require \emc{additional installation}. \emc{Crucial ones}:
\begin{description}
  \item[numpy,scipy] Numeric \& scientific libraries and linear algebra.
  \item[matplotlib] Scientific graphs and plots.
  \item[pandas] Manipulation of tabular data.
  \item[OpenCV,scikit-learn,Pytorch,Theano,TensorFlow] Machine learning.
  \item[tox, pytest, sphinx] Tests, documentation.
  \item[requests,django,flask] Web requests and web app servers.
  \item[fabric,doit] DevOps.
  \item[cffi] Bind to low-level C code.
\end{description}

\end{frame}


\begin{frame}

\frametitle{Should you take a dependency?}

Issues to consider before relying on a 3rd party library:
\begin{itemize}
\item \emc{Quality}: is it going to be any good? Good Documentation; No Bugs.
\item \emc{Community}: is it being used by many? is it being maintained? Release schedule, responsiveness to bug reports, and breaking downstream.
\item \emc{Trust}: do you believe maintainers will attack your software?
\item \emc{License}? Are you allowed to use it?
\end{itemize}

\begin{block}{}
If you are the biggest user of a library by far, you may end up having to also be the maintainer.
\end{block}

\end{frame}

\begin{frame}

\frametitle{Understanding Intellectual Property}

The state is protecting creators to incentivise investment in new inventions.
\begin{itemize}
  \item \emc{Copyright}: protects authors of artistic or literary works, including software! if you write something you (or employer) own the copyright -- can transfer it, or license it.
  \item \emc{Trade marks}: protect your trading image, name, logos, etc. to avoid brand and consumer confusion. You might have to register them.
  \item \emc{Patents}: apply and get issued one for an invention. Disclose the invention, but get protection for a limited time. Status of software patents is problematic.
\end{itemize}

\end{frame}

\begin{frame}

\frametitle{Licensing your own software, or using licensed software.}

A software license is an agreement between a creator and a user of a work, about the conditions under which the user may use the creations. 

\begin{itemize}
  \item \emc{Commercial}: creator gets some money, and allow user to use software. Usually no other rights are transfered.
  \item \emc{Free Software}: creator allows user all rights, including modifying software. But does not allow endorsement, or liability. May or not grant \emc{rights to patents}. \\
  eg.\ BSD, MIT, Apache.
  \item \emc{Gnu Public Licenses}: like free software, but \emc{viral}. Requires any modifications to also become free software under the same license. \\ eg.\ GPLv2, GPLv3, LGPL.
\end{itemize}

\emc{Check license} before using libraries!

\end{frame}


\begin{frame}

\frametitle{Philosophical discussion: what is more `free'?}

Question 1. Viral licenses impose more restrictions on users, but guarantee the resulting software is also free. Are they more or less in line with ideas of software freedom?

\vspace{5mm}

Question 2. Discuss the statement `A good non-viral library will inevitably end up being extended commercially and become closed source / non-free as a result.'

\vspace{5mm}

Question 3. What incentives are there for commercial entities to contribute to free software?

\vspace{5mm}


\end{frame}


\begin{frame}

\frametitle{The politics of intellectual `property'}

Property implies exclusive ownership, and it associated with the exercise of power, and political controversies.

\begin{itemize}
\item Is it legitimate to allow intangible `ideas' to be patented, and be owned by one party? Is this even \emc{supporting innovation}? Study controversies around software patents: \url{http://endsoftpatents.org/}
\item \emc{Digital Rights Management}: is it legitimate to enforce through software property rights? Particularly associated with intellectual property? (Remember: property is not an absolute right.)
\item Is there a right to know how your stuff works, and a \emc{right to tinker}.
\end{itemize}

\end{frame}


\begin{frame}[fragile]

\frametitle{Installing 3rd party modules}

List of great Python modules: \url{https://github.com/vinta/awesome-python}

Python Package repository, \emc{Pypi}: \url{https://pypi.python.org/pypi}

\vspace{3mm}
Python comes with \texttt{pip} as a package manager:
\begin{verbatim}
$ pip install <modulename>
\end{verbatim}
Full \texttt{pip} documentation: \url{https://pip.pypa.io/en/stable/}

\end{frame}


\begin{frame}[fragile]

\frametitle{Clashing versions of modules}

What to do if you need two different (clashing) versions of the same library?

What about testing your program or library using different library versions?

\vspace{5mm}
Solution: use \texttt{virtualenv} to create a clean virtual environment:
\begin{small}
\begin{verbatim}
$ virtualenv ENV           # Create in dir ENV
$ source ENV/bin/activate  # Activate environment
(ENV)$ pip install ...     # Install in ENV
(ENV)$ python ...          # Run in ENV
(ENV)$ deactivate          # Exit ENV
$
\end{verbatim}
\end{small}

\end{frame}


\begin{frame}

\frametitle{Make your own libraries and modules}

Why think of programs in terms of modules?
\begin{itemize}
  \item Most programs are too big to live in a single file.
  \item Clean, well defined interfaces between modules, reduce communication overheads. 
  \item Information Hiding behind modules allows for abstraction and maintainability.
  \item Controlled interactions allow decoupling of modules.
  \item High-quality, reusable code -- can be used across projects.
\end{itemize}

\end{frame}

\begin{frame}[fragile]

\frametitle{Lightweight libraries and modules}

Python allows a program to be \emc{split across many files}. 

\vspace{3mm}
Consider the two files:
\begin{verbatim}
branch.py
minimap.py
\end{verbatim}

\vspace{3mm}
Each creates a \emc{name space available to the other} through \texttt{import}. \\ Eg. in \texttt{minimap.py}:
\begin{minted}{python}
from branch import branch
\end{minted}

\end{frame}

\begin{frame}

\frametitle{From many files, to libraries, to modules \ldots}

\emc{Many files} are necessary to organize your work, within a project. Each file should contain \emc{a single concept or data type}. Directories with \emc{meaningful names} group concepts.

\vspace{3mm}
Multiple files become \emc{Libraries} once you refine \emc{clean interfaces}, \emc{few well structured dependencies} between them, and they aggressively implement \emc{abstraction and hiding}.

\vspace{3mm}
Libraries become \emc{Modules}, once you create facilities to test and \emc{distribute them independently} of the rest of your project.

\end{frame}


\begin{frame}[fragile]

\frametitle{The structure of a module}

All user-facing libraries, and Python end-user programs are usually packaged as \emc{modules}.

\vspace{3mm}
Module \texttt{<mymodule>} is a directory of files:
\begin{small}
\begin{Verbatim}[fontsize=\footnotesize]
README.rst                # Basic help on github
LICENSE                   # License
setup.py                  # Packaging & dependencies information
tox.ini                   # Test setup
euclid/__init__.py        # Modules code
euclid>/gcd.py
docs/conf.py              # Module Sphinx documentation
docs/index.rst
tests/test_gcd.py         # Module tests
\end{Verbatim}
\end{small}

See The Hitchhiker’s Guide to Python \url{http://docs.python-guide.org/en/latest/}

\end{frame}


\begin{frame}

\frametitle{The GCD function -- with a doctest}  


We define our GCD function in \texttt{euclid/gcd.py}
  \inputminted[
    xleftmargin=1.4em,
    frame=lines,
    framesep=2mm,
    %baselinestretch=1.2,
    % bgcolor=lightgray,
    fontsize=\scriptsize,
    linenos
  ]{python}{src/euclid/euclid/gcd.py}


\end{frame}

\begin{frame}

\frametitle{The module \texttt{\_\_init\_\_.py} file}

A \texttt{euclid/\_\_init\_\_.py} file indicates the directory defines a module.

\vspace{5mm}
The code in the file defines names within the module.

\vspace{5mm}
The variable \texttt{\_\_all\_\_} restricts the names exported, \\ by eg. \texttt{from euclid import *}.
\end{frame}

\begin{frame}

\frametitle{The module \texttt{\_\_init\_\_.py} file}
  
  \inputminted[
    xleftmargin=1.4em,
    frame=lines,
    framesep=2mm,
    %baselinestretch=1.2,
    % bgcolor=lightgray,
    fontsize=\scriptsize,
    linenos
  ]{python}{src/euclid/euclid/__init__.py}

\end{frame}

\begin{frame}

\frametitle{The module \texttt{setup.py}}

  \inputminted[
    xleftmargin=1.4em,
    frame=lines,
    framesep=2mm,
    %baselinestretch=1.2,
    % bgcolor=lightgray,
    fontsize=\scriptsize,
    linenos
  ]{python}{src/euclid/setup.py}

\end{frame}

\begin{frame}

\frametitle{The \texttt{setup.py} meta-data}

\begin{itemize}
  \item The \emc{name} and \emc{version} are used by pip / pypi to index your package.
  \item The \emc{packages} point to what is included in the module. \\ (Note that tests are not included in this case.)
  \item The \emc{install\_requires} lists packages required. \\ Can specify \emc{version range}. 
  \item The \emc{entry\_points} list the function that starts an application.
  \end{itemize}

\end{frame}

\begin{frame}[fragile]

\frametitle{Packaging the module}

Create a \emc{source distribution} using:
\begin{verbatim}
$ python setup.py sdist
\end{verbatim}

\vspace{3mm}
Create a \emc{binary distribution} using:
\begin{verbatim}
$ python setup.py bdist
$ python setup.py bdist --format=wininst
\end{verbatim}

\vspace{3mm}
\emc{Install manually} a package:
\begin{verbatim}
$ python setup.py install
\end{verbatim}

\end{frame}

\begin{frame}

\frametitle{Testing packaging and installation using \texttt{tox}}

\begin{itemize}
  \item Need to test packaging.
  \item Need to test install process.
  \item Need to test against many Python versions.
  \item Need to test against many library versions.
  \end{itemize}

  \vspace{5mm}
  The tool of choice for this is \texttt{tox}. You can define \emc{environments}, \emc{dependencies}, and \emc{commands} to test in them

\end{frame}

\begin{frame}[fragile]

\frametitle{The \texttt{tox.ini} file}

Test in a Python 3.5 context \ldots 
\begin{scriptsize}
\begin{verbatim}
[tox]
envlist = py35

[testenv]
deps = pytest
       pytest-cov

commands = pytest --doctest-modules -sv euclid
           pytest --cov={envsitepackagesdir}/euclid --doctest-modules -sv tests
           euclid 10 2
\end{verbatim}
\end{scriptsize}
\ldots and execute pytest, coverage and the program itself.

\end{frame}


\begin{frame}[fragile]
\begin{tiny}
\begin{verbatim}
$ tox
GLOB sdist-make: src\euclid\setup.py
py35 inst-nodeps: src\euclid\.tox\dist\euclid-0.0.1.zip
py35 installed: coverage==4.4.2, euclid==0.0.1, pytest==3.3.0, pytest-cov==2.5.1
py35 runtests: commands[0] | pytest --doctest-modules -sv euclid
============================= test session starts =============================
collecting ... collected 1 item

euclid/gcd.py::euclid.gcd.GCD PASSED                                     [100%]
========================== 1 passed in 0.03 seconds ===========================
py35 runtests: commands[1] | pytest --cov=site-packages/euclid --doctest-modules -sv tests
============================= test session starts =============================
collecting ... collected 5 items

tests/test_gcd.py::test_euclid PASSED                                    [ 20%]
tests/test_gcd.py::test_euclid_exc PASSED                                [ 40%]
tests/test_gcd.py::test_euclid_exc_raises[test_inputs0] PASSED           [ 60%]
tests/test_gcd.py::test_euclid_exc_raises[test_inputs1] PASSED           [ 80%]
tests/test_gcd.py::test_euclid_exc_raises[test_inputs2] PASSED           [100%]
----------- coverage: platform win32, python 3.5.3-final-0 -----------
Name                                             Stmts   Miss  Cover
--------------------------------------------------------------------
.tox\py35\Lib\site-packages\euclid\__init__.py      16     11    31%
.tox\py35\Lib\site-packages\euclid\gcd.py            6      0   100%
--------------------------------------------------------------------
TOTAL                                               22     11    50%
========================== 5 passed in 0.06 seconds ===========================
py35 runtests: commands[2] | euclid 10 2
2
___________________________________ summary ___________________________________
  py35: commands succeeded
  congratulations :)
\end{verbatim}
\end{tiny}

\end{frame}

\begin{frame}

\frametitle{Documentation}

Use \emc{Sphinx} and \emc{autodoc} to generate documentation. 
\begin{itemize}
  \item Include (and test) \emc{examples} in your documentation!
  \item Test your examples! (\texttt{pytest --doctest-modules -sv euclid})
  \item Provide testing documentation -- so that others can hack on your code.
\end{itemize}

\vspace{3mm}
If in doubt about documenting your function, use the \emc{Google style guide}:
\url{http://google.github.io/styleguide/pyguide.html}
\\(Document arguments, returns, exceptions, yields, examples)

\vspace{3mm}
Execute \texttt{sphinx-quickstart} to start your Sphinx documentation.
\end{frame}


\begin{frame}[fragile]

\frametitle{Documentation with \texttt{autodoc}}

Write \texttt{index.rst}. Compile with \texttt{docs/make html}:
\begin{tiny}
\begin{verbatim}
Welcome to euclid's documentation!
==================================

The euclid module provides an implementation of the Greatest Common Denominator (GCD) algorithm. 
It provides a command line utility to compute the GCD::

    $ euclid 10 2
    2

The Euclid Python module
------------------------

.. autofunction:: euclid.GCD

Development
-----------

To test the Euclid module, use tox::

    $ tox

\end{verbatim}
\end{tiny}

\end{frame}

\begin{frame}

\begin{center}
\includegraphics[scale=0.5]{assets/docs}
\end{center}

\end{frame}


\begin{frame}

\frametitle{Remember: Done Done!}

The quality installation, packaging, documentation, distinguishes the great engineer:
\begin{itemize}
  \item Always fold your projects into one or more modules.
  \item Provide packaging facilities through \texttt{setup} scrips.
  \item Allow installation through \texttt{pip}.
  \item Test in multiple environments using \texttt{tox}.
  \item Include \texttt{coverage} checking routinely in your QA process.
  \item Test packaging and deployment.
  \item Include and test documentation with \texttt{Sphinx}
  \item Upload into \texttt{pypi}, \texttt{travis}, \texttt{coveralls} and \texttt{readthedocs} \ldots
\end{itemize}

\vspace{3mm}
When learning \emc{other languages} find the \emc{equivalent facilities}.

\end{frame}

\begin{frame}

\frametitle{A note on automation: DevOps}

The software production process requires a lot of tools: interpreter, testing, building documentation, integration testing, deployment testing, coverage, linter, different environments, \ldots

\vspace{5mm}
When building network applications further automation is necessary: deploying to servers, migrating live data, complex configuration scripts, managing access control \& authentication, \ldots

\vspace{5mm}
\begin{block}{If you do it more than once, automate it!}
Any operation in relation to engineering your software, operating it or deploying it, that you \emc{perform more than once, should be automated} through Python (high-level language) scripts, and tested. \emc{DevOps} is recent tendency to merge operations and development, by \emc{automating operations aggressively}. 
\end{block}

\end{frame}

\begin{frame}

\frametitle{Key automation tools}

\begin{itemize}
\item \emc{Build tools}: keep track of changes in files, and update dependencies. Eg. \texttt{doit} library.
\item \emc{Local automation}: Define sequences of commands, on files and programs. Eg. \texttt{paver}
\item \emc{Remove management}: Define command sequences that need to be ran on remote machines, files transfered, etc. Eg. \texttt{fabric}, \texttt{ansible}.
\item \emc{Deploy isolated containers}: easily deploy whole applications, while keeping them isolated from each other. Eg. \texttt{virtualenv}, Docker, virtual machines.
  \end{itemize}

Write high-level code to automate operations; no shell scripts or repetitive manual work.

\end{frame}

\begin{frame}

\frametitle{Advanced features of the Python language}

You have now covered the basics of Python: 99\% of code is written with constructs you now know.
\begin{itemize}
  \item Class inheritance.
  \item Comprehensions.
  \item Decorators.
  \item Async / Await / Coroutines.
  \item Custom Context managers (with \ldots)
  \item Metaclasses / descriptors (meta-programming)
\end{itemize}
Some of those may change the behavior of Python, and may therefore \emc{surprise} other programmers. See the Python language reference for details: \url{https://docs.python.org/3/reference/index.html}

\end{frame}

\section{The tricky questions}

\begin{frame}

\frametitle{Question 4.1: The endless \texttt{for} loop}

We have mentioned \texttt{for} loops are safer than \texttt{while} loops. Define a name \texttt{everloop} such that the following fragment of code never terminates: 

\texttt{for \_ in everloop: pass}.

\end{frame}

\begin{frame}

\frametitle{Question 4.2: Functional programming}

\begin{itemize}
\item Write a function that takes any function. It returns a `memoized' version of it using an LRU cache. The output of the memoization function should be a function. \url{https://en.wikipedia.org/wiki/Memoization}
\item Solve the above question using a Python Decorator. \url{https://wiki.python.org/moin/PythonDecorators}
\item Ensure the docstring of the new function is the same as the docstring of the original function.
\end{itemize}

\end{frame}
 
\bibliographystyle{alpha}
\nobibliography{references}

\end{document}