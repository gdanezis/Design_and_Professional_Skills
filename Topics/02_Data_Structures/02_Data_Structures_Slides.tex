\documentclass{beamer} % [aspectratio=169]
\usetheme{ucl}
\setbeamercolor{banner}{bg=brightblue}
\setbeamersize{description width=2em}
\setbeamertemplate{navigation symbols}{\vspace{-2ex}} 

%\usepackage{fontspec}
\usepackage[utf8]{inputenc}
% \usepackage[english, greek]{babel}


\usepackage[T1]{fontenc} % Turn £ into $
\usepackage{minted}
\usemintedstyle{emacs}

\usepackage{fancyvrb}
\usepackage{xcolor}
\usepackage{url}

\usepackage{natbib}
\usepackage{bibentry}
\usepackage{url}


\usepackage{tikz}
\usetikzlibrary{positioning}



\newcommand\emc[1]{\textcolor{brightblue}{\textbf{#1}}}

\AtBeginSection[]{
  \begin{frame}
  \vfill
  \centering
  \begin{beamercolorbox}[sep=8pt,center,shadow=true,rounded=true]{title}
    \usebeamerfont{title}\insertsectionhead\par%
  \end{beamercolorbox}
  \vfill
  \end{frame}
}

\author{Prof.\ George Danezis, University College London, UK}
\title{Introduction to Data Structures and Algorithms.}
\subtitle{ENGS102P: Design and Professional Skills }
% \institute{}
\date{Term 1, 2017}


\begin{document}
\nobibliography*


\frame{
\titlepage
}

\section{Understanding Sequence Types}


\begin{frame}
\frametitle{Storing many things within a single named variable.} 

A \texttt{tuple} variable can represent a \emc{sequence of values}:

	\inputminted[
		firstline=5,
		lastline=9,
		xleftmargin=1.4em,
		frame=lines,
		framesep=2mm,
		%baselinestretch=1.2,
		% bgcolor=lightgray,
		fontsize=\footnotesize,
		linenos
	]{python}{src/core.py}


\begin{itemize}
	\item The special function \texttt{len} returns the \emc{number} of items stored.
	\item Retrive the $i^{th}$ value using \emc{indexing}: \texttt{variable[i]}.
	\item In computer science indexes start at \texttt{0}, not \texttt{1}.
	\item If the index is larger or equal to the length raises an exception.
	\item Indexing with negative numbers returns items from the end of the sequence, 
	ie.\ \texttt{numbers[-1] == 9}.
\end{itemize}

\end{frame}

\begin{frame}
\frametitle{Immutable, and mutable data types \& \texttt{list}.} 

A \texttt{tuple} variable is \emc{immutable}: it cannot be change after it is initialized.

\vspace{3mm}
A variable of type \texttt{list}, is a \emc{mutable} sequence:
	\inputminted[
		firstline=13,
		lastline=16,
		xleftmargin=1.4em,
		frame=lines,
		framesep=2mm,
		%baselinestretch=1.2,
		% bgcolor=lightgray,
		fontsize=\footnotesize,
		linenos
	]{python}{src/core.py}


\begin{block}{Why have immutable data types?}
It is easier to reason about programs in which data does not change after initialization, both for correctness; when multiple processes share the same data; and also when trying to optimize.
\end{block}

\end{frame}


\begin{frame}
\frametitle{The perils of mutability.} 

Mutable variables may have \emc{unexpected side effects}:

	\inputminted[
		firstline=24,
		lastline=30,
		xleftmargin=1.4em,
		frame=lines,
		framesep=2mm,
		%baselinestretch=1.2,
		% bgcolor=lightgray,
		fontsize=\footnotesize,
		linenos
	]{python}{src/core.py}

\vspace{3mm}
Line 30 includes into the \texttt{KL} list a \emc{reference} to the structure pointed to by \texttt{L}. It does \emc{not copy} it. When \texttt{L} changes, so does \texttt{KL}. Ummutable data types cannot surprise you in this manner.

\end{frame}


\begin{frame}
\frametitle{Heterogenous vs. Homogenous data types.}

Python allows us to include values of any \emc{mixture of types} in a sequence:
	\inputminted[
		firstline=20,
		lastline=20,
		xleftmargin=1.4em,
		frame=lines,
		framesep=2mm,
		%baselinestretch=1.2,
		% bgcolor=lightgray,
		fontsize=\footnotesize,
		linenos
	]{python}{src/core.py}

This is called a \emc{heterogenous} data structure, as compared with a \emc{homogenous} one (of a single type.)

\vspace{7mm}
Do not rely on indexing to refer to particular parts of a logical \emc{record}, it is prone to errors. Use a \texttt{NamedTuple} instead to \emc{give them good names}.

\end{frame}

\begin{frame}
\frametitle{Iterating trough sequences \& \texttt{for} control structures.}

The \texttt{for} control structure executes a block of code \emc{for all values} in a sequence in order.

	\inputminted[
		firstline=34,
		lastline=39,
		xleftmargin=1.4em,
		frame=lines,
		framesep=2mm,
		%baselinestretch=1.2,
		% bgcolor=lightgray,
		fontsize=\footnotesize,
		linenos
	]{python}{src/core.py}

For loops are \emc{safer than while} loops, since they are easier to check for termination!

\end{frame}

\section{Algorithms \& computational complexity}

\begin{frame}
\frametitle{Is a value within a sequence?}

We wish to define an algorithm \texttt{isin} to test whether a sequence contains a value.
\begin{itemize}
\item Return \texttt{True} if it is, and \texttt{False} otherwise.
\end{itemize}

\vspace{3mm}
Some \emc{simple tests} for \texttt{isin} would be:
	\inputminted[
		firstline=1,
		lastline=3,
		xleftmargin=1.4em,
		frame=lines,
		framesep=2mm,
		%baselinestretch=1.2,
		% bgcolor=lightgray,
		fontsize=\footnotesize,
		linenos
	]{python}{src/isin.py}

\end{frame}

\begin{frame}
\frametitle{Sequential search.}

	\inputminted[
		firstline=5,
		lastline=9,
		xleftmargin=1.4em,
		frame=lines,
		framesep=2mm,
		%baselinestretch=1.2,
		% bgcolor=lightgray,
		fontsize=\footnotesize,
		linenos
	]{python}{src/isin.py}

\begin{itemize}
	\item We construct a function that takes as parameters a sequence (\texttt{seq}) and value (\texttt{val}).
	\item It uses a \texttt{for} loop to sequentially go through all the values of \texttt{seq}.
	\item Within the loop each value is tested. If they match return \texttt{True}.
	\item If the loop completes, it returns \texttt{False}.
\end{itemize}

\end{frame}

\begin{frame}
\frametitle{How efficient is the \texttt{isin} algorithm?}

Assume the sequence has $n$ elements:
\begin{itemize}
\item The algorithm will iterate over $n$ elements if it returns \texttt{False}.
\item If the value is at a random position it will iterate on average over $\frac{1}{2} n$ elements.
\end{itemize}

\vspace{5mm}
The number of specific steps executed, as a function of the size of its inputs is a key measure of the \emc{time complexity} of an algorithm.

\vspace{5mm}
However, we often only care about time complexity increases (1) \emc{up to a constant} and (2) for \emc{large enough parameter sizes}.

\end{frame}

\begin{frame}
\frametitle{The $\mathcal{O}$ (Big Oh) notation.}

Lets call the average number of steps an algorithm takes $f(n)$. We define as the \emc{order of} relation between functions.

\vspace{3mm}
We say that $f(n) = \mathcal{O}(g(n))$ as $n \rightarrow \infty$ if

\[ \exists n_0, c > 0. \quad |f(n)| \leq c \cdot |g(n)| \quad \text{ for } n \geq n_0. \]

\begin{itemize}
\item In words, there exists a value $n_0$ after which the function $f(n)$ is always smaller than $g(x)$ up to a constant $c$.
\item Note that the notation \emc{hides lower order terms and constant}, \\ eg. $3n^2 + 10n + 5 = \mathcal{O}(n^2)$.
\end{itemize}

\begin{block}{}
Sequential search has time complexity in the order of $\mathcal{O}(n).$
\end{block}

\end{frame}


\begin{frame}
\frametitle{Can we search a sequence in fewer steps?}

An arbitrary sequnence cannot be searched in time less than $\mathcal{O}(n)$.

\vspace{3mm}
However, we can \emc{pre-compute an index} on the sequence:
\begin{itemize}
	\item An index is \emc{a sorted view} of a sequence that allows fast \emc{isin} operations.
\end{itemize}

\end{frame}


\begin{frame}
\frametitle{Bisection or binary search algorithm} 

\begin{itemize}
\item Define two variables \texttt{range\_start} and \texttt{range\_end} initialized with the indexes of the first and last element of the sequence + 1. 
\item Pick the middle element of the range: if it is larger than the value sought, assign its index to \texttt{range\_end}, otherwise assign it to \texttt{range\_start}. 
\item Repeat until the range is of size one.
\item If the element at the start of the range value sought return \texttt{True}. 
\item Otherwise, return \texttt{False}.
\end{itemize}

\end{frame}


\begin{frame}
\frametitle{Binary search illustrated.}

A step by step example of searching the value \texttt{17} within a sorted array.

\begin{center}
\input{assets/binarysearch.tex}
\end{center}

\vspace{5mm}
\begin{itemize}
	\item Invariants: $s$ is smaller or equal, and $e$ always beyond the index of the target.
\item Note that the last occurence of the value is found.
\end{itemize}

\end{frame}

\begin{frame}
\frametitle{The code for binary search.}

	\inputminted[
		firstline=9,
		lastline=25,
		xleftmargin=1.4em,
		frame=lines,
		framesep=2mm,
		%baselinestretch=1.2,
		% bgcolor=lightgray,
		fontsize=\footnotesize,
		linenos
	]{python}{src/binarysearch.py}

\end{frame}

\begin{frame}
\frametitle{The time complexity of binary search.}

\begin{block}{}
Binary search has time complexity in the order of $\mathcal{O}(\log n)$.
\end{block}

\vspace{3mm}
Proof. Consider the size of the variable \texttt{diff} $=n =n_0$ at step 0. We note that at each step the size of \texttt{diff} is a fraction $0 < \alpha (\approx 0.5) < 1$ of its previous size, namely $n_i = \alpha \cdot n_{i-1}$.
\begin{align}
n_i = \alpha \cdot n_{i-1} \Leftrightarrow n_i = \alpha^i \cdot n_{0}
\end{align}
After how many steps $i$ will $n_i$ become $1$, and the algorithm end?
\begin{align}
\alpha^i \cdot n_{0} = 1 \Leftrightarrow \log (\alpha^i \cdot n_{0}) &= 0 \\
i \log \alpha + \log n_{0} &= 0 \Leftrightarrow i = \left( - \frac{1}{\log \alpha} \right) \cdot \log n = \mathcal{O}(\log n).
\end{align}

\end{frame}


\begin{frame}
\frametitle{The intimate link between data structures and algorithms.}

Sequential search works on any sequence, but takes time $\mathcal{O}(n)$.

\vspace{5mm}
Binary search takes time $\mathcal{O}(\log n)$, but requires a sorted sequence. The data structure (sorted sequence) and the algorithm (binary search) work together and depend on each other.

\vspace{5mm}
\begin{block}{Is $\mathcal{O}(\log n)$ really much better than $\mathcal{O}(n)$?}
Remember $\mathcal{O}(\log n)$ is proportional to the number of binary digits in $n$ (up to a constant). So if $n=1000000$ then sequential search will take about a million steps, while binary search will take about 20. That is a big difference, and it grows as $n$ grows.
\end{block}

\end{frame}


\begin{frame}
\frametitle{Functions \& Recursion.}

A function is called \emc{recursive}, or \emc{using recursion}, if it is \emc{calling itself}!
\begin{itemize}
	\item A way of expressing a computation, possibly more naturally.
	\item Anything that can be done with recursion can be done with loops, and vice versa.
	\item Only a limited \emc{resursive depth} is allowed.
\end{itemize}

\begin{block}{What is recursion good for?}
A number of problems are naturally expressed by dividing them into \emc{sub-problems of the same form} as the initial problem. This algorithmic strategy is called \emc{divide-and-conquer} and recursion is well suited to expressing a solution.
\end{block}

\end{frame}

\begin{frame}
\frametitle{The recursive variant of binary search.}

	\inputminted[
		firstline=10,
		lastline=20,
		xleftmargin=1.4em,
		frame=lines,
		framesep=2mm,
		%baselinestretch=1.2,
		% bgcolor=lightgray,
		fontsize=\footnotesize,
		linenos
	]{python}{src/binarysearch_recurse.py}

\begin{itemize}
	\item Recursion at line 18 and 20.
	\item Other tricks: default parameters (line 10) and conditional expression (line 11).
\end{itemize}

\end{frame}

\section{Sorting}

\begin{frame}
\frametitle{Sorting.}

Binary search, and a number of other efficient algorithms, rely on sorted sequences. 

\vspace{3mm}
We next study:
\begin{itemize}
	\item A sorting algorithm, \emc{merge sort}.
	\item The \emc{time complexity} of sorting.
	\item How to show sorting is \emc{correct}.
	% \item Issues around maintaining \emc{large sorted indexes}.
	\item How to write \emc{generic} sorting and searching algorithms.
\end{itemize}

\end{frame}

\begin{frame}
\frametitle{Merge sort (I).}

Mergesort was proposed by John von Neumann in 1945.

\vspace{5mm}
Algorithm 1. Merge two sorted sequences into one sorted sequence:
\begin{itemize}
\item While both sequence are not empty, consider the first unprocessed item from both.
\item Add to the front of the new sequence the smaller or available one, and remove it from its sequence. Repeat.
\item Append remaining elements of one of the lists.
\end{itemize}

\end{frame}

\begin{frame}
\frametitle{Merge sort (II).}

Algorithm 2. Merge Sort.
\begin{itemize}
\item Sequences of length up to one are sorted!
\item Divide into two subsequences of equal length.
\item Merge Sort both sequences separately.
\item Merge the two sorted sequences using Algorithm 1.
\end{itemize}

\end{frame}



\begin{frame}
\frametitle{Tests for sorting.}

	\inputminted[
		% firstline=10,
		lastline=14,
		xleftmargin=1.4em,
		frame=lines,
		framesep=2mm,
		%baselinestretch=1.2,
		% bgcolor=lightgray,
		fontsize=\footnotesize,
		linenos
	]{python}{src/mergesort.py}

\end{frame}

\begin{frame}
\frametitle{Merge two sorted lists (algorithm 1).}

	\inputminted[
		firstline=25,
		lastline=41,
		xleftmargin=1.4em,
		frame=lines,
		framesep=2mm,
		%baselinestretch=1.2,
		% bgcolor=lightgray,
		fontsize=\footnotesize,
		linenos
	]{python}{src/mergesort.py}

\end{frame}

\begin{frame}
\frametitle{A recursive implementation of mergesort  (algorithm 2).}

	\inputminted[
		firstline=17,
		lastline=23,
		xleftmargin=1.4em,
		frame=lines,
		framesep=2mm,
		%baselinestretch=1.2,
		% bgcolor=lightgray,
		fontsize=\footnotesize,
		linenos
	]{python}{src/mergesort.py}

\begin{itemize}
\item Python \emc{slicing}: \texttt{items[start:end]} returns the list from index \texttt{start} up to, but not including, index \texttt{end}. If omitted from zero to the length of the list.
\item Use \texttt{items.copy()} to get a new \emc{copy} of the list (remember mutability bugs).
\end{itemize}

\end{frame}

\begin{frame}
\frametitle{How to argue mergesort is correct.}

Structural induction (verbal reasoning): 
\begin{itemize}
	\item Argue that the merge operation is correct: given two sorted lists it produces a sorted list.
	\item Base case of mergesort: a single element list is sorted.
	\item Inductive case of mergesort: assume mergesort for a previous step works, prove that the result is sorted. Follows from the properties of merge.
\end{itemize}

\vspace{3mm}
Note that the recursive strucutre of the program supports the argument of correctness, through induction.

\end{frame}

\begin{frame}
\frametitle{What is the cost of mergesort.}

The time complexity of merge (algorithm 1) is $\mathcal{O}(n)$, where $n$ is the sum of the lengths of both lists.

\vspace{3mm}

Mergesort (algorithm 2) can only divide an array of $n$ elements $\mathcal{O}(\log n)$ times. At each level of subdivision the overall costs of the merge operations will be $\mathcal{O}(n)$ (total from algorithm 1).

\vspace{3mm}
Therefore the overall time complexity of mergesort is of the \emc{order $\mathcal{O}(n \log n)$}. 

\vspace{3mm}
This is \emc{optimal} for comparison based algorithms (but note that \emc{constant factors may vary}.) Space complexity may also vary.

\end{frame}

\section{A note on debugging}

\begin{frame}
\frametitle{The art of debugging.}

\begin{block}{What is debugging?}
Debugging is the \emc{structured and systematic} approach taken to identify the \emc{root cause} of an error in a program, and fix it in the code.
\end{block}

\begin{itemize}
	\item Debugging necessitates \emc{having detected} a fault. Ensure great tests and coverage to ensure you detects bugs early, to detect when they are likely fixed.
	\item Debugging small fragments of code that do one simple thing is easier than large complex programs. \emc{Write and run tests for every few lines of code}, to detect errors early.
\end{itemize}

\end{frame}

\begin{frame}
\frametitle{Debugging requires understanding.}

What to check for:
\begin{itemize}
	\item Do you \emc{understand the algorithm} you are using, and how / why it works?
	\item Do you \emc{understand why the error} appears?
	\item Can you \emc{identify the simplest case} in which an erroneous result is returned? (Write a test case!)
\end{itemize}

\begin{block}{Prof.\ Mark Handley's advice}
The bug is likely in your head, before it appears in your code. \emc{Do not attempt to modify your program} until the answer to those is `yes'.
\end{block}

\end{frame}

\begin{frame}
\frametitle{How to understand your program.}

The scientific method, applied:
\begin{itemize}
	\item Use the \emc{simplest case} you have observed your program fails on.
	\item \emc{Predict} the behavior of your program, ie.\ the values of all variables, and control flow to be followed.
	\item Confirm through \emc{empirical observation} that the variable values and control flow is as predicted.
\end{itemize}

\vspace{3mm}
Gaps between your prediction and your observation indicate bugs in either, or both in, your thinking or your program. Both need fixing.

\end{frame}

\begin{frame}
\frametitle{Tools to observe programs.}

\begin{itemize}
\item Write additional simpler or more specific \emc{test cases} and observe how they fail. \url{https://docs.pytest.org/en/latest/}
\item Use the Python \emc{logging} libraries and facilities to log the values of variables. Do not leave debugging \texttt{print} statements in production code. \url{https://docs.python.org/3/howto/logging.html}.
\item Use a \emc{tracer} to dump all the steps taken by your program on a simple failing test case. \url{https://docs.python.org/3.0/library/trace.html}
\item Use the Python \emc{debugger} to single step through the execution of the program, and observe the variable values. \url{https://docs.python.org/3.2/library/pdb.html}
\end{itemize}

\end{frame}

\section{More sequences and generic programming}

\begin{frame}
\frametitle{Strings and byte sequences.}

Besides \emc{lists} and \emc{tuples} Python supports a number of other sequence types.
\begin{itemize}
	\item \emc{strings} (\texttt{str}): represent sequences of unicode characters.
	\item \emc{bytes} (\texttt{bytes}): represent a sequences of raw bytes (ie. values from 0 to 255).
	\item Strings are transformed into and from sequences of bytes through an \emc{encoding} such as \texttt{utf8} or \texttt{ASCII}.
\end{itemize}

	\inputminted[
		firstline=3,
		lastline=5,
		xleftmargin=1.4em,
		frame=lines,
		framesep=2mm,
		%baselinestretch=1.2,
		% bgcolor=lightgray,
		fontsize=\footnotesize,
		linenos
	]{python}{src/strbytes.py} 

\end{frame}

\begin{frame}
\frametitle{Internationalization (i18n).}

Old encodings such as ASCII only map latin characters to bytes. Others cannot exist!

\vspace{3mm}
Software and services need to be \emc{accessible to all people in their native languages}. Always use string representations that can represent all languages (and emoji!) such as \emc{unicode}, and encodings into bytes such as \emc{utf-8} to represent those strings as bytes.

\begin{itemize}
\item Do not be a biggot and assume English as the only possible language.
\item Plan for internationalization: do not hard code english strings into software, but rather load them from a file that can be localized.
\item Dates, currency, number representations also vary per locale.
\end{itemize}

\end{frame}

\begin{frame}
\frametitle{Operations on string.}

Strings support a number of operations:
	\inputminted[
		firstline=7,
		lastline=17,
		xleftmargin=1.4em,
		frame=lines,
		framesep=2mm,
		%baselinestretch=1.2,
		% bgcolor=lightgray,
		fontsize=\footnotesize,
		linenos
	]{python}{src/strbytes.py} 

For full reference see: \url{https://docs.python.org/3.1/library/string.html}

\end{frame}

\begin{frame}
\frametitle{String literrals \& formatting.}

\begin{itemize}
\item String constants in programs can be encosed in single or double quotes.
\item You can represent special characters in strings using the \texttt{`\textbackslash'} symbol, such as new line as \texttt{{\textbackslash}n} and tab as \texttt{{\textbackslash}t}.
\item The \texttt{format} method substitutes into the string:
\end{itemize}

	\inputminted[
		firstline=19,
		lastline=25,
		xleftmargin=1.4em,
		frame=lines,
		framesep=2mm,
		%baselinestretch=1.2,
		% bgcolor=lightgray,
		fontsize=\footnotesize,
		linenos
	]{python}{src/strbytes.py} 


\end{frame}

\begin{frame}
\frametitle{Generic programming: how to sort a list of strings.}

Do you need to \emc{re-implement mergesort} for lists of strings? Remember the \emc{abstraction principle}.

\vspace{5mm}
However our implementation of mergesort seems to just work:
	\inputminted[
		firstline=29,
		lastline=33,
		xleftmargin=1.4em,
		frame=lines,
		framesep=2mm,
		%baselinestretch=1.2,
		% bgcolor=lightgray,
		fontsize=\footnotesize,
		linenos
	]{python}{src/strbytes.py}

\emc{Generic programming} ensures that algorithms are independent of and agnostic about the exact types they are processing.


\end{frame}

\begin{frame}
\frametitle{Why generic programming works.}

Merge (algorithm 1) only uses `\texttt{<=}' on items within the list to sort:
	\inputminted[
		firstline=30,
		lastline=36,
		xleftmargin=1.4em,
		frame=lines,
		framesep=2mm,
		highlightlines={31},
		%baselinestretch=1.2,
		% bgcolor=lightgray,
		fontsize=\footnotesize,
		linenos
	]{python}{src/mergesort.py}

If the operation greater-or-equal (compare) is supported correctly by the type of object in the list, mergesort will return the correct result.

\end{frame}

\begin{frame}
\frametitle{Sorting strings by length, rather than lexicographically.}

Strategy: allow the programmer to specify a function that performs a comparison. Parametrize mergesort by the comparison function.

	\inputminted[
		firstline=20,
		lastline=26,
		xleftmargin=1.4em,
		frame=lines,
		framesep=2mm,
		highlightlines={20,24-26},
		%baselinestretch=1.2,
		% bgcolor=lightgray,
		fontsize=\footnotesize,
		linenos
	]{python}{src/mergesort_cmp.py}



\end{frame}

\begin{frame}
% \frametitle{Sorting strings by length, rather than lexicographically.}

	\inputminted[
		firstline=28,
		lastline=44,
		xleftmargin=1.4em,
		frame=lines,
		framesep=2mm,
		highlightlines={34},
		%baselinestretch=1.2,
		% bgcolor=lightgray,
		fontsize=\footnotesize,
		linenos
	]{python}{src/mergesort_cmp.py}


\end{frame}

\begin{frame}
\frametitle{Example of different comparison functions for strings.}

	\inputminted[
		firstline=46,
		lastline=55,
		xleftmargin=1.4em,
		frame=lines,
		framesep=2mm,
		fontsize=\footnotesize,
		linenos
	]{python}{src/mergesort_cmp.py}

\end{frame}

\begin{frame}
\frametitle{Functions are just variables.}

\begin{block}{Functions are first-class objects.}
In modern programming languages functions are first-class objects (of type \texttt{function}). They are assigned to names, and they can be passed as arguments to function calls; stored in data structures; etc. 
\end{block}

The shorthand \texttt{lambda} notation can be used to define functions within expressions.
	\inputminted[
		firstline=57,
		lastline=62,
		xleftmargin=1.4em,
		frame=lines,
		framesep=2mm,
		fontsize=\footnotesize,
		linenos
	]{python}{src/mergesort_cmp.py}


\end{frame}

\section{Persistent storage}

\begin{frame}
\frametitle{Persistent storage with files.}

\begin{itemize}
\item \emc{File systems} provide persistent storage, and expose it through a hierarchy of \emc{directories} and \emc{files}.
\item A file is a \emc{sequence of bytes} written to persistent memory (disk or other).
\item Files support \emc{read} and \emc{write} operations on arbitrary positions of this sequence, as well as \emc{updating} and \emc{appending} data at the end of files.
\end{itemize}

Python supports handling files in \texttt{text mode} or \texttt{binary mode}. The first takes care of encoding of stings semi-automatically, while the second is more flexible but requires manual conversion of everything to bytes.

\end{frame}

\begin{frame}
\frametitle{The open function.}

The \texttt{open} function takes as parameters a file path and a mode, and returns a file object that may be used to \texttt{read} or \texttt{write}. All files should be \emc{closed} after processing.

\vspace{3mm}
Modes include a combination of text (`t') or binary (`b'); read (`r'), write and truncate (`w') or append (`a'); and optionally update (`+').

\vspace{3mm}
The full documentation for \texttt{open}: \url{https://docs.python.org/3/library/functions.html\#open}

\end{frame}

\begin{frame}
\frametitle{Reading all data from a text file.}

	\inputminted[
		firstline=3,
		lastline=7,
		xleftmargin=1.4em,
		frame=lines,
		framesep=2mm,
		fontsize=\footnotesize,
		linenos
	]{python}{src/open.py}

Calling \texttt{read()} reads the full contents of the file into memory. Remember to \emc{close files} to free system resources.


\end{frame}

\begin{frame}
\frametitle{Reading and writing text files.}

	\inputminted[
		firstline=11,
		lastline=20,
		xleftmargin=1.4em,
		frame=lines,
		framesep=2mm,
		fontsize=\footnotesize,
		linenos
	]{python}{src/open.py}

\begin{itemize}
	\item Opening a file with `w' \emc{truncates} the file. (`r+' to update; `a' to append.)
	\item A file is an \emc{iterator}. The \texttt{for} loop reads each line in turn.
	\item Closing files can get tedious.
\end{itemize}


\end{frame}

\begin{frame}
\frametitle{The `with' control structure.}

\emc{Acquiring and freeing} resources is a very common pattern. The with control structure removes the need for manual closing of files, or freeing resources.

	\inputminted[
		firstline=22,
		lastline=29,
		xleftmargin=1.4em,
		frame=lines,
		framesep=2mm,
		fontsize=\footnotesize,
		linenos
	]{python}{src/open.py}

\end{frame}

\begin{frame}
\frametitle{Working with binary files.}

You can use \emc{seek} and \emc{tell} to select a position in the file. 
	\inputminted[
		firstline=31,
		lastline=39,
		xleftmargin=1.4em,
		frame=lines,
		framesep=2mm,
		fontsize=\footnotesize,
		linenos
	]{python}{src/open.py}


\end{frame}

\begin{frame}[fragile]
\frametitle{Standard input and output.}

Programs have a standard input and standard output that act as files for reading and writing data.

	\inputminted[
		firstline=43,
		lastline=47,
		xleftmargin=1.4em,
		frame=lines,
		framesep=2mm,
		fontsize=\footnotesize,
		linenos
	]{python}{src/open.py}

	\inputminted[
		firstline=58,
		lastline=59,
		xleftmargin=1.4em,
		frame=lines,
		framesep=2mm,
		fontsize=\footnotesize,
		linenos
	]{python}{src/open.py}


\begin{Verbatim}
$ echo "Hello" | python src/open.py
HELLO
\end{Verbatim}

\end{frame}


\begin{frame}
\frametitle{How to test programs that access the environment.}

`\emc{Monkeypatching}' is a test tactic in which you \emc{substitute functions that access the environment} with known `dummy' ones.

	\inputminted[
		firstline=49,
		lastline=56,
		xleftmargin=1.4em,
		frame=lines,
		framesep=2mm,
		fontsize=\footnotesize,
		linenos
	]{python}{src/open.py}

You can monkey patch to \emc{inject known data} pretending to be the environment. Or to \emc{test interactions} with the environment. Tests should not access the environment.

\end{frame}


\begin{frame}
\frametitle{Dealing with a lot of data.}

\emc{Big data}: Some processing tasks involve datasets that are \emc{larger than the working memory} of a single computer.
\begin{itemize}
\item Reading the full file in memory is not possible.
\item Instead \emc{on-line algorithms} must be implemented to \emc{work incrementally}.
\item Read small chunks of data; process; write small chunks of output.
\end{itemize}

\begin{block}{Sorting files on disk}
Early computers had a tiny amount of working memory, and used tapes as persistent storage. Mergesort was especially important since it may be used to sort files in an incremental way with access to very little memory (see Exercise 2-3).
\end{block}

\end{frame}

\begin{frame}
\frametitle{The social importance of sorting and indexing.}

Modern search engines, need to find information within milliseconds!

\vspace{3mm}
Fast information retrieval is achieved through \emc{indexing}:
\begin{itemize}
	\item The file to be indexed is scanned linearly, and a list of tuples of (word, byte position) are extracted.
	\item The tuples are then sorted to produce an \emc{inverted index}.
	\item The inverted index is stored to disk alongside the file.
	\item To find all instances of a term binary search is used on the inverted index to \emc{retrieve the positions} in the file, if any, the term appears.
\end{itemize}

The second ingredient of search engines is \emc{ranking}, that selects which files are most relevant for a search term.
\end{frame}

\begin{frame}
\frametitle{Searching, Indexing, Files \& Privacy.}

Data have \emc{meaning}, and can impact people's lives.\\
Protecting \emc{people's rights}, including \emc{privacy}, part of professional conduct.

\vspace{3mm}
UK Data Protection Act and the new General Data Protection Regulations:
\begin{itemize}
	\item Defines `personal information' as anything relating to a living individual.
	\item Lawful grounds (consent, contract, law) must exist to process.
	\item Can only store for defined purposes, time frames.
	\item Must be secured.
	\item Rights: access, remedy, to be forgotten.
\end{itemize}

{\small \url{https://ico.org.uk/for-organisations/guide-to-data-protection/}}

\end{frame}


\begin{frame}
\frametitle{Technical implications of regulations.}

\begin{itemize}
	\item All personal information is mutable.
	\item Need to be able to find and remove it \\ (Right to be Forgotten, Subject Access).
	\item That includes indexes and file backups.
	\item Notices and privacy statements need to be provided.
	\item No data processing should take place beyond what was notified.
	\item You need to use appropriate security controls.
\end{itemize}
\ldots 4th year courses COMPGA17 (Privacy Enhancing Technolgies) and COMPGA01/02 (Computer Security 1 \& 2).

\end{frame}

\section{Exercises}

\begin{frame}
\frametitle{Exercise 2-1 Removing recursion.}

Recursion is one strategy for implementing algorithms. However, it might be useful to implement algorithms without recursion in some cases.
\begin{description}
	\item[2-1.1] What resources does recursion use, and in what cases would the use of recursion exhaust them.
	\item[2-1.2] Refactor the \texttt{mergesort} algorithm provided to remove recursion (hint: use while loop, and a list to store tasks that would otherwise be deal with a recursive call.)
	\item[2-1.3] Use a profiler to observe the differences between the recursive implementation and your implementation. What are the differences?
\end{description}

\end{frame}

\begin{frame}
\frametitle{Exercise 2-2 Adding in recursion.}

The implementation of \texttt{isin\_bisect} presented does not use recursion. However it could.
\begin{description}
	\item[2-2.1] Refactor the \texttt{isin\_bisect} function to use recursion.
	\item[2-2.2] Argue why your implementation is correct.
	\item[2-3.2] Discuss whether the existing tests for the function are appropriate for the recursive implementation, and justify any modifications necessary.
\end{description}

\end{frame}

\begin{frame}
\frametitle{Exercise 2-3 Sorting large files.}

Download the classic novel `War and Peace' by Leo Tolstoy from Project Gutenberg: \url{https://www.gutenberg.org/files/2600/2600-0.txt}
\begin{description}
	\item[2-3.1] Write a variant of the \texttt{merge} function that given two files, with lines sorted lexicographically, creates a new file containing the sorted lines of both files.
	\item[2-3.2] Ensure and argue that the working memory your \texttt{merge} function uses is constant in relation to the size of any files involved (ie.\ $\mathcal{O}(1)$).
	\item[2-3.3] Implement a variant of \texttt{mergesort} that takes a text file, and produces another text file containing the sorted lines of the first one. (Hint: Split the file into multiple ones and use \texttt{merge}).
	\item[2-3.4] Ensure and argue that the total working memory of your \texttt{mergesort} is $\mathcal{O}(1)$, and produce a sorted version of `War and Peace'.
\end{description}

\end{frame}

\begin{frame}
\frametitle{Exercise 2-4 Searching large sorted files.}

We studied binary search on an array. However if a data file is too large, you cannot read all contents in the working memory.
\begin{description}
	\item[2-4.1] Write a function that takes a file and a byte position, and returns the string that is contained in this byte position. (Hint: careful about encodings.)
	\item[2-4.2] Refactor the \texttt{isin\_bisect} function to take a text file parameter and string, and return the whether the string appears as a line in the file using $\mathcal{O}(1)$ working memory only.
	\item[2-4.3] Discuss the complexities in the implementation, and possible strategies with regards to storing sorted files to overcome them.
\end{description}

\end{frame}

% ---------------------------------

\bibliographystyle{alpha}
\nobibliography{references}

\end{document}