\documentclass{beamer} % [aspectratio=169]
\usetheme{ucl}
\setbeamercolor{banner}{bg=brightblue}
\setbeamersize{description width=2em}
\setbeamertemplate{navigation symbols}{\vspace{-2ex}} 

\usepackage[T1]{fontenc} % Turn £ into $
\usepackage{minted}
\usemintedstyle{emacs}

\usepackage{fancyvrb}
\usepackage{xcolor}
\usepackage{url}

\usepackage{natbib}
\usepackage{bibentry}
\usepackage{url}

\usepackage{tikz}
\usetikzlibrary{positioning}

\newcommand\emc[1]{\textcolor{brightblue}{\textbf{#1}}}

\AtBeginSection[]{
  \begin{frame}
  \vfill
  \centering
  \begin{beamercolorbox}[sep=8pt,center,shadow=true,rounded=true]{title}
    \usebeamerfont{title}\insertsectionhead\par%
  \end{beamercolorbox}
  \vfill
  \end{frame}
}

\author{Prof.\ George Danezis \\ University College London, UK}
\title{Introduction to Data Structures and Algorithms.}
\subtitle{ENGS102P: Design and Professional Skills }
% \institute{}
\date{Term 1, 2017}


\begin{document}
\nobibliography*


\frame{
\titlepage
}

\section{Understanding Sequence Types}


\begin{frame}
\frametitle{Storing many things within a single named variable.} 

A \texttt{tuple} variable can represent a \emc{sequence of values}:

	\inputminted[
		firstline=5,
		lastline=9,
		xleftmargin=1.4em,
		frame=lines,
		framesep=2mm,
		%baselinestretch=1.2,
		% bgcolor=lightgray,
		fontsize=\footnotesize,
		linenos
	]{python}{src/core.py}


\begin{itemize}
	\item The special function \texttt{len} returns the \emc{number} of items stored.
	\item Retrive the $i^{th}$ value using \emc{indexing}: \texttt{variable[i]}.
	\item In computer science indexes start at \texttt{0}, not \texttt{1}.
	\item If the index is larger or equal to the length raises an exception.
	\item Indexing with negative numbers returns items from the end of the sequence, 
	ie.\ \texttt{numbers[-1] == 9}.
\end{itemize}

\end{frame}

\begin{frame}
\frametitle{Immutable, and mutable data types \& \texttt{list}.} 

A \texttt{tuple} variable is \emc{immutable}: it cannot be change after it is initialized.

\vspace{3mm}
A variable of type \texttt{list}, is a \emc{mutable} sequence:
	\inputminted[
		firstline=13,
		lastline=16,
		xleftmargin=1.4em,
		frame=lines,
		framesep=2mm,
		%baselinestretch=1.2,
		% bgcolor=lightgray,
		fontsize=\footnotesize,
		linenos
	]{python}{src/core.py}


\begin{block}{Why have immutable data types?}
It is easier to reason about programs in which data does not change after initialization, both for correctness; when multiple processes share the same data; and also when trying to optimize.
\end{block}

\end{frame}


\begin{frame}
\frametitle{The perils of mutability.} 

Mutable variables may have \emc{unexpected side effects}:

	\inputminted[
		firstline=24,
		lastline=30,
		xleftmargin=1.4em,
		frame=lines,
		framesep=2mm,
		%baselinestretch=1.2,
		% bgcolor=lightgray,
		fontsize=\footnotesize,
		linenos
	]{python}{src/core.py}

\vspace{3mm}
Line 30 includes into the \texttt{KL} list a \emc{reference} to the structure pointed to by \texttt{L}. It does \emc{not copy} it. When \texttt{L} changes, so does \texttt{KL}. Ummutable data types cannot surprise you in this manner.

\end{frame}


\begin{frame}
\frametitle{Heterogenous vs. Homogenous data types.}

Python allows us to include values of any \emc{mixture of types} in a sequence:
	\inputminted[
		firstline=20,
		lastline=20,
		xleftmargin=1.4em,
		frame=lines,
		framesep=2mm,
		%baselinestretch=1.2,
		% bgcolor=lightgray,
		fontsize=\footnotesize,
		linenos
	]{python}{src/core.py}

This is called a \emc{heterogenous} data structure, as compared with a \emc{homogenous} one (of a single type.)

\vspace{7mm}
Do not rely on indexing to refer to particular parts of a logical \emc{record}, it is prone to errors. Use a \texttt{NamedTuple} instead to \emc{give them good names}.

\end{frame}

\begin{frame}
\frametitle{Iterating trough sequences \& \texttt{for} control structures.}

The \texttt{for} control structure executes a block of code \emc{for all values} in a sequence in order.

	\inputminted[
		firstline=34,
		lastline=39,
		xleftmargin=1.4em,
		frame=lines,
		framesep=2mm,
		%baselinestretch=1.2,
		% bgcolor=lightgray,
		fontsize=\footnotesize,
		linenos
	]{python}{src/core.py}

For loops are \emc{safer than while} loops, since they are easier to check for termination!

\end{frame}

\section{Algorithms \& computational complexity}

\begin{frame}
\frametitle{Is a value within a sequence?}

We wish to define an algorithm \texttt{isin} to test whether a sequence contains a value.
\begin{itemize}
\item Return \texttt{True} if it is, and \texttt{False} otherwise.
\end{itemize}

\vspace{3mm}
Some \emc{simple tests} for \texttt{isin} would be:
	\inputminted[
		firstline=1,
		lastline=3,
		xleftmargin=1.4em,
		frame=lines,
		framesep=2mm,
		%baselinestretch=1.2,
		% bgcolor=lightgray,
		fontsize=\footnotesize,
		linenos
	]{python}{src/isin.py}

\end{frame}

\begin{frame}
\frametitle{Sequential search.}

	\inputminted[
		firstline=5,
		lastline=9,
		xleftmargin=1.4em,
		frame=lines,
		framesep=2mm,
		%baselinestretch=1.2,
		% bgcolor=lightgray,
		fontsize=\footnotesize,
		linenos
	]{python}{src/isin.py}

\begin{itemize}
	\item We construct a function that takes as parameters a sequence (\texttt{seq}) and value (\texttt{val}).
	\item It uses a \texttt{for} loop to sequentially go through all the values of \texttt{seq}.
	\item Within the loop each value is tested. If they match return \texttt{True}.
	\item If the loop completes, it returns \texttt{False}.
\end{itemize}

\end{frame}

\begin{frame}
\frametitle{How efficient is the \texttt{isin} algorithm?}

Assume the sequence has $n$ elements:
\begin{itemize}
\item The algorithm will iterate over $n$ elements if it returns \texttt{False}.
\item If the value is at a random position it will iterate on average over $\frac{1}{2} n$ elements.
\end{itemize}

\vspace{5mm}
The number of specific steps executed, as a function of the size of its inputs is a key measure of the \emc{time complexity} of an algorithm.

\vspace{5mm}
However, we often only care about time complexity increases (1) \emc{up to a constant} and (2) for \emc{large enough parameter sizes}.

\end{frame}

\begin{frame}
\frametitle{The $\mathcal{O}$ (Big Oh) notation.}

Lets call the average number of steps an algorithm takes $f(n)$. We define as the \emc{order of} relation between functions.

\vspace{3mm}
We say that $f(n) = \mathcal{O}(g(n))$ as $n \rightarrow \infty$ if

\[ \exists n_0, c > 0. \quad |f(n)| \leq c \cdot |g(n)| \quad \text{ for } n \geq n_0. \]

\begin{itemize}
\item In words, there exists a value $n_0$ after which the function $f(n)$ is always smaller than $g(x)$ up to a constant $c$.
\item Note that the notation \emc{hides lower order terms and constant}, \\ eg. $3n^2 + 10n + 5 = \mathcal{O}(n^2)$.
\end{itemize}

\begin{block}{}
Sequential search has time complexity in the order of $\mathcal{O}(n).$
\end{block}

\end{frame}


\begin{frame}
\frametitle{Can we search a sequence in fewer steps?}

An arbitrary sequnence cannot be searched in time less than $\mathcal{O}(n)$.

\vspace{3mm}
However, we can \emc{pre-compute an index} on the sequence:
\begin{itemize}
	\item An index is \emc{a sorted view} of a sequence that allows fast \emc{isin} operations.
\end{itemize}

\end{frame}


\begin{frame}
\frametitle{Bisection or binary search algorithm} 

\begin{itemize}
\item Define two variables \texttt{range\_start} and \texttt{range\_end} initialized with the indexes of the first and last element of the sequence + 1. 
\item Pick the middle element of the range: if it is larger than the value sought, assign its index to \texttt{range\_end}, otherwise assign it to \texttt{range\_start}. 
\item Repeat until the range is of size one.
\item If the element at the start of the range value sought return \texttt{True}. 
\item Otherwise, return \texttt{False}.
\end{itemize}

\end{frame}


\begin{frame}
\frametitle{Binary search illustrated.}

A step by step example of searching the value \texttt{17} within a sorted array.

\begin{center}
\input{assets/binarysearch.tex}
\end{center}

\vspace{5mm}
\begin{itemize}
	\item Invariants: $s$ is smaller or equal, and $e$ always beyond the index of the target.
\item Note that the last occurence of the value is found.
\end{itemize}

\end{frame}

\begin{frame}
\frametitle{The code for binary search.}

	\inputminted[
		firstline=9,
		lastline=25,
		xleftmargin=1.4em,
		frame=lines,
		framesep=2mm,
		%baselinestretch=1.2,
		% bgcolor=lightgray,
		fontsize=\footnotesize,
		linenos
	]{python}{src/binarysearch.py}

\end{frame}

\begin{frame}
\frametitle{The time complexity of binary search.}

\begin{block}{}
Binary search has time complexity in the order of $\mathcal{O}(\log n)$.
\end{block}

\vspace{3mm}
Proof. Consider the size of the variable \texttt{diff} $=n =n_0$ at step 0. We note that at each step the size of \texttt{diff} is a fraction $0 < \alpha (\approx 0.5) < 1$ of its previous size, namely $n_i = \alpha \cdot n_{i-1}$.
\begin{align}
n_i = \alpha \cdot n_{i-1} \Leftrightarrow n_i = \alpha^i \cdot n_{0}
\end{align}
After how many steps $i$ will $n_i$ become $1$, and the algorithm end?
\begin{align}
\alpha^i \cdot n_{0} = 1 \Leftrightarrow \log (\alpha^i \cdot n_{0}) &= 0 \\
i \log \alpha + \log n_{0} &= 0 \Leftrightarrow i = \left( - \frac{1}{\log \alpha} \right) \cdot \log n = \mathcal{O}(\log n).
\end{align}

\end{frame}


\begin{frame}
\frametitle{The intimate link between data structures and algorithms.}

Sequential search works on any sequence, but takes time $\mathcal{O}(n)$.

\vspace{5mm}
Binary search takes time $\mathcal{O}(\log n)$, but requires a sorted sequence. The data structure (sorted sequence) and the algorithm (binary search) work together and depend on each other.

\vspace{5mm}
\begin{block}{Is $\mathcal{O}(\log n)$ really much better than $\mathcal{O}(n)$?}
Remember $\mathcal{O}(\log n)$ is proportional to the number of binary digits in $n$ (up to a constant). So if $n=1000000$ then sequential search will take about a million steps, while binary search will take about 20. That is a big difference, and it grows as $n$ grows.
\end{block}

\end{frame}


\begin{frame}
\frametitle{Functions \& Recursion.}

A function is called \emc{recursive}, or \emc{using recursion}, if it is \emc{calling itself}!
\begin{itemize}
	\item A way of expressing a computation, possibly more naturally.
	\item Anything that can be done with recursion can be done with loops, and vice versa.
	\item Only a limited \emc{resursive depth} is allowed.
\end{itemize}

\begin{block}{What is recursion good for?}
A number of problems are naturally expressed by dividing them into \emc{sub-problems of the same form} as the initial problem. This algorithmic strategy is called \emc{divide-and-conquer} and recursion is well suited to expressing a solution.
\end{block}

\end{frame}

\begin{frame}
\frametitle{The recursive variant of binary search.}

	\inputminted[
		firstline=10,
		lastline=20,
		xleftmargin=1.4em,
		frame=lines,
		framesep=2mm,
		%baselinestretch=1.2,
		% bgcolor=lightgray,
		fontsize=\footnotesize,
		linenos
	]{python}{src/binarysearch_recurse.py}

\begin{itemize}
	\item Recursion at line 18 and 20.
	\item Other tricks: default parameters (line 10) and conditional expression (line 11).
\end{itemize}

\end{frame}


\begin{frame}
\frametitle{Sorting.}

Binary search, and a number of other efficient algorithms, rely on sorted sequences. 

\vspace{3mm}
We next study:
\begin{itemize}
	\item A sorting algorithm, \emc{merge sort}.
	\item The \emc{time complexity} of sorting.
	\item How to show sorting is \emc{correct}.
	\item Issues around maintaining \emc{large sorted indexes}.
	\item How to write \emc{generic} sorting and searching algorithms.
\end{itemize}

\end{frame}

\begin{frame}
\frametitle{Merge sort.}

Algorithm 1. Merge two sorted sequences into one sorted sequence:
\begin{itemize}
\item While both sequence are not empty, consider the first unprocessed item from both.
\item Add to the front of the new sequence the smaller or available one, and remove it from its sequence. Repeat.
\item Append remaining elements of one of the lists.
\end{itemize}

Algorithm 2. Merge Sort.
\begin{itemize}
\item Sequences of length up to one are sorted!
\item Divide into two subsequences of equal length.
\item Merge Sort both sequences separately.
\item Merge the two sorted sequences using Algorithm 1.
\end{itemize}

\end{frame}


\begin{frame}
\frametitle{Tests for sorting.}

	\inputminted[
		% firstline=10,
		lastline=14,
		xleftmargin=1.4em,
		frame=lines,
		framesep=2mm,
		%baselinestretch=1.2,
		% bgcolor=lightgray,
		fontsize=\footnotesize,
		linenos
	]{python}{src/mergesort.py}

\end{frame}

\begin{frame}
\frametitle{Merge two sorted lists (algorithm 1).}

	\inputminted[
		firstline=25,
		lastline=41,
		xleftmargin=1.4em,
		frame=lines,
		framesep=2mm,
		%baselinestretch=1.2,
		% bgcolor=lightgray,
		fontsize=\footnotesize,
		linenos
	]{python}{src/mergesort.py}

\end{frame}

\begin{frame}
\frametitle{A recursive implementation of mergesort  (algorithm 2).}

	\inputminted[
		firstline=17,
		lastline=23,
		xleftmargin=1.4em,
		frame=lines,
		framesep=2mm,
		%baselinestretch=1.2,
		% bgcolor=lightgray,
		fontsize=\footnotesize,
		linenos
	]{python}{src/mergesort.py}

\begin{itemize}
\item Python \emc{slicing}: \texttt{items[start:end]} returns the list from index \texttt{start} up to, but not including, index \texttt{end}. If omitted from zero to the length of the list.
\item Use \texttt{items.copy()} to get a new \emc{copy} of the list (remember mutability bugs).
\end{itemize}

\end{frame}


\begin{frame}
\frametitle{The art of debugging.}

\begin{block}{What is debugging?}
Debugging is the \emc{structured and systematic} approach taken to identify the \emc{root cause} of an error in a program, and fix it in the code.
\end{block}

\begin{itemize}
	\item Debugging necessitates \emc{having detected} a fault. Ensure great tests and coverage to ensure you detects bugs early, to detect when they are likely fixed.
	\item Debugging small fragments of code that do one simple thing is easier than large complex programs. \emc{Write and run tests for every few lines of code}, to detect errors early.
\end{itemize}

\end{frame}

\begin{frame}
\frametitle{Debugging requires understanding.}

What to check for:
\begin{itemize}
	\item Do you \emc{understand the algorithm} you are using, and how / why it works?
	\item Do you \emc{understand why the error} appears?
	\item Can you \emc{identify the simplest case} in which an erroneous result is returned? (Write a test case!)
\end{itemize}

\begin{block}{Prof.\ Mark Handley's advice}
The bug is likely in your head, before it appears in your code. \emc{Do not attempt to modify your program} until the answer to those is `yes'.
\end{block}

\end{frame}

\begin{frame}
\frametitle{How to understand your program.}

The scientific method, applied:
\begin{itemize}
	\item Use the \emc{simplest case} you have observed your program fails on.
	\item \emc{Predict} the behavior of your program, ie.\ the values of all variables, and control flow to be followed.
	\item Confirm through \emc{empirical observation} that the variable values and control flow is as predicted.
\end{itemize}

\vspace{3mm}
Gaps between your prediction and your observation indicate bugs in either, or both in, your thinking or your program. Both need fixing.

\end{frame}

\begin{frame}
\frametitle{Tools to observe programs.}

\begin{itemize}
\item Write additional simpler or more specific \emc{test cases} and observe how they fail. \url{https://docs.pytest.org/en/latest/}
\item Use the Python \emc{logging} libraries and facilities to log the values of variables. Do not leave debugging \texttt{print} statements in production code. \url{https://docs.python.org/3/howto/logging.html}.
\item Use a \emc{tracer} to dump all the steps taken by your program on a simple failing test case. \url{https://docs.python.org/3.0/library/trace.html}
\item Use the Python \emc{debugger} to single step through the execution of the program, and observe the variable values. \url{https://docs.python.org/3.2/library/pdb.html}
\end{itemize}

\end{frame}

\section{Stuff}

\begin{frame}[fragile]
\frametitle{Using Python interactively.} 

\begin{itemize}
	\item Install Python (\url{https://www.python.org/about/gettingstarted/})
	\item Open a command line console, and run the Python interpreter.
	\item Type your fist command: \mint{Python}{print("Hello World!")}
	\item You should see it executing!

\begin{Verbatim}[fontsize=\footnotesize]
george$ python3
Python 3.5.2 (default, Nov 17 2016, 17:05:23) 
[GCC 5.4.0 20160609] on linux
Type "help", "copyright", "credits" or "license" for more information.
>>> print("Hello World!")
Hello World!
>>>
\end{Verbatim}

\end{itemize}

\end{frame}


% ---------------------------------

\bibliographystyle{alpha}
\nobibliography{references}

\end{document}