\documentclass{beamer} % [aspectratio=169]
\usetheme{ucl}
\setbeamercolor{banner}{bg=brightblue}
\setbeamersize{description width=2em}
\setbeamertemplate{navigation symbols}{\vspace{-2ex}} 

%\usepackage{fontspec}
\usepackage[utf8]{inputenc}
% \usepackage[english, greek]{babel}


\usepackage[T1]{fontenc} % Turn £ into $
\usepackage{minted}
\usemintedstyle{emacs}

\usepackage{fancyvrb}
\usepackage{xcolor}
\usepackage{url}

\usepackage{natbib}
\usepackage{bibentry}
\usepackage{url}


\usepackage{tikz}
\usetikzlibrary{positioning}



\newcommand\emc[1]{\textcolor{brightblue}{\textbf{#1}}}

\AtBeginSection[]{
  \begin{frame}
  \vfill
  \centering
  \begin{beamercolorbox}[sep=8pt,center,shadow=true,rounded=true]{title}
    \usebeamerfont{title}\insertsectionhead\par%
  \end{beamercolorbox}
  \vfill
  \end{frame}
}

\author{Prof.\ George Danezis, University College London, UK}
\title{Dynamic Data Structures \& Objects Oriented Concepts.}
\subtitle{ENGS102P: Design and Professional Skills }
% \institute{}
\date{Term 1, 2017}


\begin{document}
\nobibliography*


\frame{
\titlepage
}

\begin{frame}
\frametitle{How are the python dynamic data types implemented?}

Dynamic data types are those, like \emc{list}, that can grow and shrink in the number of elements that contain, and maintain some \emc{invariants}, such as being sorted.

\vspace{3mm}
But \emc{how can you implement} lists or other such datatypes?

\vspace{3mm}
\begin{block}{Understand how things work!}
Great computer scientists use the \emc{best available libraries and native types}, but understand how they work sufficiently to make \emc{informed decisions} about their applicability to problems. When they are not applicable they know how to \emc{build their own}.
\end{block}

\end{frame}

\begin{frame}
\frametitle{Aims for this topic}

\begin{itemize}
	\item Understand how \emc{dynamic data types are implemented} \\ under the hood.
	\item Explore in details how to implement \emc{linked lists}.
	\item Introduce \emc{Object Oriented Programming} \\ to make your own data types look and feel like native ones.
	\item Implement a data type that maintains a \emc{sorted multi-set} \\ of items using a tree.
	\item Discuss the native \emc{dictionary type} (\texttt{dict}).
	\item Introduce the concept of \emc{Machine learning}, \\ and implement a decision forest for a task.
\end{itemize}

\end{frame}


\section{Defining your own data types}

\begin{frame}
\frametitle{The linked list data type.}
\end{frame}

\begin{frame}
\frametitle{Simple operations on linked list.}
\end{frame}

\begin{frame}
\frametitle{Linked list algorithms.}
\end{frame}

\begin{frame}
\frametitle{Testing linked lists.}
\end{frame}


\begin{frame}
\frametitle{The structure of the linked list code.}

 Sources of error.
\end{frame}

\section{Object Oriented Programming}

\begin{frame}
\frametitle{Objects = Data + Algorithms.}
\end{frame}

\begin{frame}
\frametitle{Classes, objects / instances and methods.}
\end{frame}

\begin{frame}
\frametitle{The Class LinkedList.}
\end{frame}

\begin{frame}
\frametitle{The concept of Encapsulation and Hiding.}
\end{frame}

\begin{frame}
\frametitle{The importance of immutability.}
\end{frame}

\begin{frame}
\frametitle{Overloading operators for seamless coding.}
\end{frame}

\begin{frame}
\frametitle{Iterators.}
\end{frame}

\section{Keeping a multi-set sorted.}

\begin{frame}
\frametitle{Motivation: keeping indexes for dynamic data structures.}
\end{frame}

\begin{frame}
\frametitle{Trees and sorted trees.}
\end{frame}

\begin{frame}
\frametitle{Trees as branches and Leaves.}
\end{frame}

\begin{frame}
\frametitle{The TreeNode Class.}
\end{frame}

\begin{frame}
\frametitle{Adding items.}
\end{frame}

\begin{frame}
\frametitle{Removing items.}
\end{frame}

\begin{frame}
\frametitle{Searching.}
\end{frame}

\begin{frame}
\frametitle{Traversing in order.}
\end{frame}

\begin{frame}
\frametitle{Computational complexity.}
\end{frame}

\section{The Python \texttt{dict} type.}

\section{From programming to learning.}

\section{Ethics, Machine Learning and Artificial Intelligence.}

% ---------------------------------

\bibliographystyle{alpha}
\nobibliography{references}

\end{document}